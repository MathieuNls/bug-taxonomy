
\section{Dicussion}

In this section, we discuss the answers of our three research questions.

\subsection{RQ$_1$: What are the proportions of different types of
bugs?}


One important finding of this study is that there is significantly more Types 3 and 4 bugs than Types 1 and 2 in all studied systems.
Moreover, this observation is not system-specific.
The traditional one-bug/one-fault (i.e. Type 1) way of thinking about bugs only accounts for 6.8\%of the bugs.

We believe that, triaging algorithms \cite{Jalbert2008,Jeong2009,Khomh2011a,Tamrawi2011a} can benefit from these findings by developing techniques that can detect Type 2 and 4 bugs.
This would result in better performance in terms of reducing the cost, time and efforts required by the developers in the bug fixing process.

\subsection{RQ$_2$: How complex is each type of bugs?}

To evaluate the complexity of each types of bug, we have computed five process metrics (time to close, duplications, reopenings, comments and severity) and four code metrics (files, commit, hunks and churns).

{\bf Process complexity}: For four out the five process metrics we used, we found that types 3 and 4 combined performed significantly worst than Types 1 and 2.
The only process metric where types 3 and 4 do not performed significantly worst than types 1 and 2 is the severity.
Although clear guidelines exist on how to assign the severity of a bug, it remains a manual process done by the bug reporter.
In addition, previous studies, notably those by Khomh et al.
\cite{Khomh2011a}, showed that severity is not a consistent/trustworthy characteristic of a BR, which lead to he emergence of studies for predicting the severity of bugs (e.g., \cite{Lamkanfi2010,Lamkanfi2011,Tian2012}).
Nevertheless, we discovered that, in our ecosystems, types 3 and 4 are  have an higher severity than types 1 and 2.

{\bf Code complexity}: All our code metrics (files, commit, hunks and churns) are showing similar results.
Indeed, in all cases, Types 3 and 4 perform worst than Types 1 and 2, suggesting, once again, that Types 3 and 4 are, in fact, more complex than Types 1 and 2.

While current approaches aiming to predict which bug will be reopen use the amount of modified files \cite{Shihab2010,Zimmermann2012,Lo2013}, we believe that they can be improved by taking into account the type of a the bug.
For example, if we can detect that an incoming bug if of Type 3 or 4 then it is more likely to reopened than a bug of Type 1 or 2.
Similarly, approaches aiming to predict the files in which a given bug should be fixed could be categorized and improved by knowking the bug type in advance \cite{Zhou2012,Kim2013a}.
Similarly to reopening, we believe that approaches targeting the identification of
duplicates \cite{Bettenburg2008a,Jalbert2008,Sun2010,Tian2012a}  could leverage this taxonomy to
achieve even better performances in terms of recall and
precision.
Finally, we believe that, approaches aiming to predict the fixing
time of a bug (e.g., \cite{Panjer2007,Bhattacharya2011,Zhang2013}) can highly benefit from
accurately predicting the type of a bug and thereforebetter
plan the required man-power to fix the bug.


\subsection{RQ$_3$: How pertinent is a bug taxonomy?}
