
\section{Dicussion}

{\bf Repartition of bug types}: One important finding of this
study is that there is significantly more Types 2 and 4 bugs
than Types 1 and 3 in all studied systems. Moreover, this
observation is not system-specific. The traditional one-bug/
one-fault way of thinking about bugs only accounts for 35\%
of the bugs. We believe that, recent triaging algorithms
\cite{Jalbert2008,Jeong2009,Khomh2011a,Tamrawi2011a} can benefit from these findings by developing
techniques that can detect Type 2 and 4 bugs. This would
result in better performance in terms of reducing the cost,
time and efforts required by the developers in the bug fixing
process.

{\bf Severity of bugs}: We discussed the severity and the
complexity of a bug in terms of its likelihood to be reopened
or marked as duplicate (RQ2). Although clear guidelines exist
on how to assign the severity of a bug, it remains a manual
process done by the bug reporter. In addition, previous
studies, notably those by Khomh et al. \cite{Khomh2011a}, showed that severity is not a consistent/trustworthy characteristic of a BR,
which lead to he emergence of studies for predicting the
severity of bugs (e.g., \cite{Lamkanfi2010,Lamkanfi2011,Tian2012}). Nevertheless, we
discovered that there is a significant difference between the
severities of Types 1 and 3 compared to Types 2 and 4.

{\bf Complexity of bugs}: At the complexity level, we use the
number of times a bug is reopened as a measure of
complexity. Indeed, if a developer is confident enough in
his/her fix to close the bug and that the bug gets reopened it
means that the developer missed some dependencies of the
said bug or did not foresee the consequences of the fix.
We found that there is a significant relationship between
the number of reopenings and type of a bug. In other words,
there is a significant relationship between the complexity and
the type of a given bug. In our datasets, Types 1 and 3 bugs
are reopened in 1.88\% of the cases, while Types 2 and 4 are
reopened in 5.73\%. Assuming that the reopening is a
representative metric for the complexity of bug, Types 2 and
4 are three times more complex than Types 1 and 3. Finally, if
we consider multiple reopenings, Types 2 and 4 account for
almost 80\% of the bugs that reopened more than once and
more than 96\% of the bug opened more than twice.
While current approaches aiming to predict which bug
will be reopen use the amount of modified files \cite{Shihab2010,Zimmermann2012,Lo2013}, we
believe that they can be improved by taking into account the type of a the bug. For example, if we can detect that an
incoming bug if of Type 2 or 4 then it is more likely to
reopened than a bug of Type 1 or 3. Similarly, approaches
aiming to predict the files in which a given bug should be
fixed could be categorized and improved by knowking the
bug type in advance \cite{Zhou2012,Kim2013a}.

{\bf Impact of a bug}: To measure the impact of bugs in end-users
and developers, we use the number of times a bug is
duplicated. This is because if a bug has many duplicates, it
means that a large number of users have experienced and a
large number of developers are blocked the failure.
We found that there is a significant relationship between
the bug type and the fact that it gets duplicated. Types 1 and 3
bugs are duplicated in 1.41\% of the cases while Types 2 and 4
are duplicated in 3.14\%. Assuming that the amount of
duplication is an accurate metric for the impact of bug, Types
2 and 4 have more than two times bigger impact than Types 1
and 3. Similarly to reopening, if we consider multiple
duplication, Types 2 and 4 account for 75\% of the bugs that
get duplicated more than once and more than 80\% of the bugs
that get duplicated more than twice.
We believe that approaches targeting the identification of
duplicates \cite{Bettenburg2008a,Jalbert2008,Sun2010,Tian2012a}  could leverage this taxonomy to
achieve even better performances in terms of recall and
precision.

{\bf Fixing time}: Our third research question aimed to determine
if the type of a bug impacts its fixing time. Not only we found
that the type of a bug does significantly impact its fixing time,
but we also found that, in average Types 2 and 4, stay open
111.26 days while Types 1 and 3 last for 77.36 days. Types 2
and 4 are 1.4 time longer to fix than Types 1 and 3.We
therefore believe that, approaches aiming to predict the fixing
time of a bug (e.g., \cite{Panjer2007,Bhattacharya2011,Zhang2013}) can highly benefit from
accurately predicting the type of a bug and thereforebetter
plan the required man-power to fix the bug.
In summary, Types 2 and 4 account for 65\% of the bugs
and they are more complex, have a bigger impact and take
longer to be fixed than Types 1 and 3 while being equivalent
in terms of severity.

Our taxonomy aimed to analyse: (1) the
proportion of each type of bugs; (2) the complexity of each
type in terms of severity, reopening and duplication; (3) the
required time to fix a bug depending on its type. The key
findings are:
\begin{itemize}
  \item Types 2 and 4 account for 65\% of the bugs.
  \item Types 2 and 4 have a similar severity compared to
Types 1 and 3.
  \item Types 2 and 4 are more complex (reopening) and have
a bigger impact (duplicate) than Types 1 and 3.
  \item It takes more time to fix Types 2 and 4 than Types 1
and 3.
\end{itemize}

Our taxonomy and results can be built upon in order to classify
past and new researches in several active areas such as bug
reproduction and triaging, prediction of reopening,
duplication, severity, files to fix and fixing time. Moreover, if
one could predict the type of a bug at submission time, all
these areas could be improved.
